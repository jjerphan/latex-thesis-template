
\makeglossaries

\newglossaryentry{latex}
{
    name=latex,
    description={Is a mark up language specially suited
    for scientific documents}
}

\newglossaryentry{maths}
{
    name=mathematics,
    description={Mathematics is what mathematicians do}
}

\newglossaryentry{arr}
{
    name=arr,
    description={Annual Recurring Revenue — Revenu récurrent annuel : un
indicateur pour mesurer les revenus dans un système économique basé sur une
base continue, plutôt que sur un achat unique. Les revenus récurrents annuels
correspondraient à vos revenus récurrents mensuels multipliés par douze mois.
Tout comme ces revenus, l'ARR ne comprend que des frais d'abonnement ou des
frais récurrents engagés et fixes. \todo[inline]{continuer}}
}

\newglossaryentry{dask}
{
    name=dask,
    description={Dask}
}

\newglossaryentry{git}
{
    name=git,
    description={Un système de versionnage mis au point par Linus Torvald.
C'est de loin le logiciel le plus utilisé pour la collaboration autour de
projets informatique. Des alternatives existent comme SVN ou encore Mercurial.}
}

\newglossaryentry{sklearn}
{
    name=sklearn,
    description={Scikit-Learn, une librairie populaire pour l'apprentissage
automatique}
}

\newglossaryentry{python}
{
    name=python,
    description={python}
}

\newglossaryentry{frontend}
{
    name=frontend,
    description={frontend}
}

\newglossaryentry{issue}
{
    name=issue,
    description={issue}
}

\newglossaryentry{pullrequest}
{
    name=pullrequest,
    description={pull request}
}

\newglossaryentry{uml}
{
    name=uml,
    description={uml}
}

